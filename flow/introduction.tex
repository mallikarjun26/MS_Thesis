Motivation and Objectives
-------------------------

General face problems
Fundamental problems to solve those
Problems I choose

Explain in detail how fid solves each of the general problem
- Biometrics
    Face recognition
        Aids in aligning
    Expression detection
        Structural information
    Gender detection
        Structural statistics

Explain in detail how front solves each of the general problem
- Animation
    Frontalization
    Mimicry
    
Computer vision as a field includes making inference out of images. There has been decades of research 
happening in this field. We also have seen fruitful applications achieved in various domains. Be it 
reconstructing of 3D world which helps in automatic car navigation, analysis of medical images for 
early detection of diseases, analysis of astronimical images to study universe in general, biometric
systems for security purposes using finger print or face images. Even though there has been decades of 
research, problems mentioned above are not completely solved as there is demand for more precision
and efficient solutions. Technological advancement from other ends such as camera capablilites,
computational capacity aids in solving problems which were not feasible earlier.

In this thesis, we are going to concentrate on approaches and applications on face images. We have seen
a lot of generic problems such as face detection, face recognition, expression detection, gaze
identification, face renactment etc currently being under research and also deployed in various 
systems. Most of the above system depends on pre-processing steps such as representation, face fiducial 
detection, tranformation etc which influences the effectiveness to a great extent. Specifically in this
thesis we consider the problem of face fiducial detection and face frontalization whose approach are 
used as pre-processing steps.

Consider a basic face recognition system. To have the representation of an individual's face as a 
model, we need to be sure the representation of face has to be aligned to get a meaningful model. 
Face fiducial detection is used as a pre-processing step to extract the features at the corresponding
locations while building the model. Similarly for an expression detection system, statistical 
distribution facial landmarks with each other helps in predecting accurate expression. Even the 
gender detection system would need facial landmarks as their initial point as fiducials are in
general differently distributed for different gender.

Other pre-processing step which we discuss in this thesis is face frontalization. Given a profile
view face, a frontal view face is synthesized. This approach helps in improving the accuracy of 
face recognition system as the profile view faces would result in degraded performance. It is
also used for face renactment where video of a person can be mimicked by replacing the face with
another.

Methodology
-----------

- Quote and explanation
    Q
    
    Cant handle all variations
    Modifying the model isn't efficient
    
- Fiducial detection
    No models to select the best!
    Exemplars for the rescue.
    
- Frontalization
    No estimation of 3D models
    Just a 2D association to find the nearest exemplar


“Represent all the data with a nonparametric model rather than trying to
summarize it with a parametric model, because with very large data sources, the
data holds a lot of detail... Now go out and gather some data, and see what it can
do.”
Alon Halevy, Peter Norvig, and Fernando Pereira,
“The Unreasonable Effectiveness of Data” (Google, 2009)

The above statement is more so true when the data at hand is diverse at many fronts. In our case, 
the data is the face images which can have variations in terms of age, gender, expression, pose, 
facial structure etc. Trying to come up with a parametric model which holds all these information
leads to poorly performing systems. Also whenever there is new data, the model has to be updated
which is not so efficient. Rather its better to go with the data driven solutions.

For face fiducial detection, we employ exemplar based approach to to select the best solution 
from among outputs of regression and mixture of trees based algorithms (which we call candidate 
algorithms). We show that by using a very simple SIFT and HOG based descriptor, it is possible to 
identify the most accurate fiducial outputs from a set of results produced by candidate algorithms 
on any given test image. 

For face frontalization We employ an exemplar based approach to find the transformation that relates 
the profile view to the frontal view, and use it to generate realistic frontalizations. Our method 
does not involve estimating 3D model of the face, which is a common approach in previous work in 
this area. This leads to an efficient solution, since we avoid the complexity of adding one more 
dimension to the problem. 

Contributions and Novelties
---------------------------
