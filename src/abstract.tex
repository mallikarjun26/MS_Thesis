% Exemplar based approaches have been proved to work in diverse problems such as object detection [],
% image impainting [], object removal [], action recognition [], gesture recognition []. Exemplar 
% based methods directly utilizes the information residing in the examples to achieve a certain 
% objective, instead of trying to come up with a model representing all the examples and has shown 
% to be effective. In this thesis, we propose exemplar based approach for both face fiducial 
% detection and face frontalization.


Computer vision systems concerning face images play a vital role in security and other commercial applications. 
There has been decades of research work done by the vision community in solving the fundamental problems
such as face fiducial detection, face recognition, gender detection etc. Solving these problems
leads to interesting applications in surveillance, animation industries. 

In this thesis, we consider two fundamental problems. First, we discuss the problem of face fiducial
detection and secondly the problem of face frontalization. Exemplar based approaches have been proved 
to work in diverse problems such as object detection~\cite{malisiewicz-iccv11}, image impainting~\cite{Criminisi:2004:RFO:2319036.2320602}, 
object removal~\cite{conf/cvpr/CriminisiPT03}, action 
recognition~\cite{weinland:inria-00544741}, gesture recognition~\cite{Shen:2012:DHG:2206425.2206457}. 
These approaches directly utilizes the information residing in the examples to achieve a certain 
objective, instead of coming up with a model representing all the examples and has shown to be effective. 
In this thesis, we propose exemplar based approach for both face fiducial detection and face frontalization.

Face fiducial detection involves detecting key points on the faces such as eye corner, nose tip, mouth tips
etc. Fiducial detection is one of the main pre-processing step done for face recognition, animation, 
gender detection, gaze identification and expression recognition systems.
Number of different approaches like active shape models~\cite{milborrowCVPR08_ASM}, regression based 
methods~\cite{yuECCV14_CoR}, cascaded neural networks~\cite{zhangECCV14_deepfacealign}, tree based methods
~\cite{xhuCVPR12_wild} and exemplar based approaches~\cite{kumarPAMI13_faceExem} have been proposed in the 
recent past. Many of these algorithms only address part of the problems in this area. We propose an 
exemplar based approach which takes adavantage of the complimentarity of different approaches and achieve
consistently superior performance over the state-of-the-art methods. We provide extensive experiments 
over three popular datasets.

Face frontalization is the process of synthesizing frontal view of the face given a non-frontal view. 
Frontalization is used in intelligent photo editing tools and also aids in improving the accuracy of face
 recognition systems. Methods previously proposed involve estimating the 3D model or assuming a generic
3D model of the face. Esitmating an accurate 3D model of the face is not a completely solved problem and 
assumption of generic 3D model of the face results in loss of crucial shape cues. We propose an exemplar
based approach which doesn't involve estimating a 3D model. We show that our method performs consistently
better than the other approach considered and also efficient.
