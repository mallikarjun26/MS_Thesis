% Exemplar based approaches have been proved to work in diverse problems such as object detection [],
% image impainting [], object removal [], action recognition [], gesture recognition []. Exemplar 
% based methods directly utilizes the information residing in the examples to achieve a certain 
% objective, instead of trying to come up with a model representing all the examples and has shown 
% to be effective. In this thesis, we propose exemplar based approach for both face fiducial 
% detection and face frontalization.

% face detection
% face recognition
% facial reenactment
% facial expression analysis
% gender detection

Computer vision solutions such as face detection and recognition, facial reenactment, facial expression analysis
and gender detection have seen fruitful applications in various domains such as security, surveillance, social media 
and animation. Many of the above solutions have common pre-processing steps such as fiducial detection, appearance
modeling, face structural modelings \etc. These steps can be considered as fundamental problems to be solved in building any
computer vision solutions concerning face images.

% Computer vision systems concerning face images play a vital role in security and other commercial applications. 
% There has been decades of research work done by the vision community in solving the fundamental problems
% such as face fiducial detection, face recognition, gender detection etc. Solving these problems
% leads to interesting applications in surveillance and animation industries. 

In this thesis, we propose exemplar based approaches to solve two fundamental problems, such as face fiducial 
detection and face frontalization. Exemplar based approaches have been proved 
to work in various computer vision problems, such as object detection, image impainting, 
object removal, action 
recognition, gesture recognition. 
This approach directly utilizes the information residing in the examples to achieve a certain 
objective, instead of coming up with a model representing all the examples and has shown to be effective. 

Face fiducial detection involves detecting key points on the faces such as eye corner, nose tip, mouth tips
\etc. It is one of the main pre-processing step done for face recognition, facial animation, 
gender detection, gaze identification and expression recognition systems.
Number of different approaches like active shape models, regression based 
methods, cascaded neural networks, tree based methods
 and exemplar based approaches have been proposed in the 
recent past. Many of these algorithms only address part of the problems in this area. We propose an 
exemplar based approach which takes advantage of the complimentarity of different approaches and achieve
consistently superior performance over the state-of-the-art methods. We provide extensive experiments 
over three popular datasets.

Face frontalization is the process of synthesizing frontal view of the face given a non-frontal view. 
Method proposed for frontalization can be used in intelligent photo editing tools and also aids in improving the accuracy of face
 recognition systems. Methods previously proposed involve estimating the 3D model or assuming a generic
3D model of the face. Estimating an accurate 3D model of the face is not a completely solved problem and 
assumption of generic 3D model of the face results in loss of crucial shape cues. We propose an exemplar
based approach which does not require 3D model of the face. We show that our method is efficient and performs consistently
better than other approaches.
