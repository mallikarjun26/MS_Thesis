Once the frontal view of the nearest exemplar is obtained, we need to transform the input profile
face to a frontal view. To do this, we first \emph{transfer} correspondences between the exemplar
pairs to the input image. This is done by replacing positions of the profile exemplar landmarks with
those of the input image.
Using the landmarks
obtained, we define around 110 triangles on the face, each of which can be considered as a plane in
the face coordinate system. Since the triangles are defined based on particular set of landmarks, we
have correspondences between planes in the input image and corresponding frontal view
exemplar. We obtain the affine transformations between the corresponding planes and  
then synthesize the frontal view of the input image using these transformations.

 Figure~\ref{fig:method_pipeline} pictorially represents our method. 
For a given profile view input face, $A_p$, we retrieve most similar exemplar, $I^i_p$ from the $D$
along with its corresponding frontal view face, $I^i_f$. We then compute affine
transformations, $H^i$, to transform planes of $A_p$ to generate its frontal view.
