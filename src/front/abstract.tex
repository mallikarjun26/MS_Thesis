Face \emph{ frontalization} is the process of synthesizing a frontal view of a face, given its non-frontal view.
Frontalization is used in intelligent photo editing tools and also aids in improving the accuracy of 
face recognition systems.
For example, in the case of photo editing, faces of persons in a group photo can be
corrected to look into the camera, if they are looking elsewhere.
Similarly, even though recent methods in face recognition claim accuracy 
which surpasses that of humans in some cases, performance of 
recognition systems degrade when profile view of faces are given as input. One
way to address this issue is to synthesize frontal views of faces before recognition.

We propose a simple and efficient method to address the face frontalization problem. Our method leverages the fact that 
faces in general have a definite structure and can be represented in a low dimensional subspace. We employ an exemplar 
based approach to find the transformation that relates the profile view to the frontal view, and use
it to generate realistic frontalizations.
Our method does not involve estimating {\sc 3d} model of the face, which is a common approach in previous
work in this area. This leads to an efficient solution, since we avoid the complexity 
of adding one more dimension to the problem. Our method also retains the structural information of the individual as compared to 
that of a recent method~\cite{DBLP:journals/corr/HassnerHPE14}, which assumes a generic {\sc 3d} model for synthesis. 
We show impressive qualitative and quantitative results in comparison to the state-of-the-art in
this field.

