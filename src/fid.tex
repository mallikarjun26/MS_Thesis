\section{Abstract}
Facial fiducial detection is a challenging problem for several reasons
 like varying pose, appearance, expression, partial occlusion and others.
In the past, several approaches like mixture of trees~\cite{xhuCVPR12_wild}, regression based
methods~\cite{artizzzuICCV13_COFW},  
exemplar based methods~\cite{kumarPAMI13_faceExem} have been proposed to tackle this challenge.

In this paper, we propose an exemplar based approach to select the best solution from among outputs of
regression and mixture of trees based algorithms (which we call candidate algorithms). 
We show that by using a very simple SIFT and
HOG based descriptor, it is possible to identify the most accurate
fiducial outputs from a set of results produced by candidate
algorithms on any given test image. Our approach manifests as two algorithms, one 
based on optimizing an objective function with quadratic terms and the other based
on simple kNN. Both algorithms take as input
fiducial locations produced by running state-of-the-art
candidate algorithms on an input image, and output
accurate fiducials using a set of automatically selected exemplar images with annotations.
Our surprising result is that in this case, a simple algorithm like kNN is able to take
advantage of the seemingly huge complementarity of these candidate algorithms,
better than optimization based algorithms.

We do extensive experiments on several datasets, and show that our approach
outperforms state-of-the-art consistently. In some cases, we report as much
as a $10\%$ improvement in accuracy. We also extensively analyze each
component of our approach, to illustrate its efficacy.

An implementation and extended technical report of our approach is available www.sites.google.com/site/wacv2016facefiducialexemplars.

\section{Related Work}

\section{Face Fiducial Detection}

\subsection{Formulation}

\subsection{Algorithm Outline}

\subsection{Exemplar Selection}

\subsection{Output selection by kNN}

\subsection{Output selection by Optimization}

\subsection{Implementation Details}

\section{Results}

\subsection{Qualitative Results}

\subsection{Experimental Analysis}

\section{References}
