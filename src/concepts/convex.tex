% Linear programming
% Quadratic programming
% Quadratically constrained quadratic programming
% Integer Programming
% Integer relaxation

Optimization in mathematical sense is to select a particular sample out of all 
possible samples which yields best solution in some sense. Based on the sample
spaces and the type of function which defines the outcome, we can categorize the
optimization problem in various types. We describe few types of optimization and 
its solutions which are relevant to this thesis in the following section.

\textbf{Linear optimization} is special case of mathematical optimization in which the
solution space is defined by the linear equality and inequality constraints with
a linear objective function. The solution space is a convex polytope. Linear
optimization problem can be canonically represented as

\begin{equation}
O(x) = \arg\max_x(c^Tx)
\end{equation}
subjected to $Ax \leq b$ and $x \geq 0$. Where $x$ is the variable vector, $A$ is a
matrix and $b$ is vector which defines the solution space. There are various
algorithms like \textit{Simplex algorithm, Criss-Cross algorithm, Interior 
point method}, which can solve for global optimum.

\textbf{Quadratic programming} is another special case of mathematical optimization
with a quadratic objective function subjected to linear constraints. It can be
formulated as follows,

\begin{equation}
O(x) = \arg\max_x{(\frac{1}{2}x^TQx + c^Tx)}
\end{equation}
subjected to $Ax \leq b$. $Q$ is real symmetric matrix. Similar to linear programming,
there are many algorithms like \textit{Interior point, gradient projection, conjugate
gradient} to solve general problems.

\textbf{Qudratically constrained quadratic program} is another case similar to
Quadratic programming, but with both objective and constraints are quadratic functions.

In this thesis, we end up with a linear objective function with quadratic constraints
and solutions need to be integers. Since it can not be solved in polynomial time, we 
aim to get approximate solution with integer relaxation and solve it as linear 
optimization problem.
