\label{sec:related_work}

In this section, we categorize recent facial fiducial detection algorithms and discuss their advantages in brief.

\textbf{Active Appearance Models (AAM):}
The AAM framework has existed for almost two decades \cite{edwa_tayl_coot1998,Baker03} and the traditional AAM based
methods have not been suitable for fiducial detection \emph{in the wild} \cite{Saragih_PR2009,gross2005generic}. However, some recent methods that deviate from the traditional 
pixel-value based texture model have shown new promise \cite{antonakos2014hog,asthana_pami_2015}.   

\textbf{Constrained Local Models (CLMs):}
The CLM framework has existed for a decade \cite{Cristinacce2006,Saragih2011_1} and has been shown
to be more
capable of handling \emph{in the wild} settings. In short, CLM is a part-based approach that relies on the
locally trained detectors to generate response maps for each fiducial point followed by a simple Gauss-Newton
method based optimization \cite{Saragih2011_1} for facial shape estimation. A regression based strategy for CLM
optimization has also been proposed recently \cite{akshay_CVPR_2013}.

\textbf{Exemplar Methods:}
Exemplar based approaches have been popular since Belhumeur \etal \textquotesingle s
work~\cite{kumarPAMI13_faceExem}. Zhao \etal~\cite{zhaoACCV12_AAM} use gray scale pixel values and HOG features
to select k-nearest neighbor training faces, from which they construct a target-specific AAM at
runtime. Smith \etal~\cite{smithCVPR14_Context} and Shen \etal~\cite{shenCVPR13_retrieval} perform Hough voting
using k-NN exemplar faces, which provides
robustness to variations in appearance due to occlusion, illumination and expression. Finally,
Zhou \etal~\cite{zhouICCV13_EGM} combine an exemplar-based approach with graph-matching for robust facial fiducial 
localization. Since, we build upon outputs of candidate algorithms, we take inherent advantage
of the shape based regularization schemes employed by individual approaches and thus either side-step
this problem (section~\ref{subsec:output_selection}) or \emph{smoothen} candidate outputs using
optimization (Figure~\ref{fig:with_without_structural_costs}) in our algorithms.

\textbf{Cascaded Regression Based Methods:}
Cascaded regression based methods are considered to be the current state-of-the-art for facial fiducial detection
\cite{xiongCVPR13_SDM,asthanaCVPR14_Chehra,6909614,Tzimiropoulos_2015_CVPR,CaoWWS12}. All these methods are capable for 
robustly handling \emph{in the wild} settings in real-time. In general, the training strategy is to synthetically perturb each of the ground truth
shapes and extract robust image features (SIFT or HOGs) around each of the perturbed fiducial points. The regression
is then used to learn a mapping from these features to the shape perturbation w.r.t the ground truth shape.
Generally, a cascaded regression based strategy is adopted to learn this mapping and has been shown to converge
in 4-5 iterations \cite{xiongCVPR13_SDM,asthanaCVPR14_Chehra}.  

 A recent work of Smith \etal ~\cite{smithECCV14_ED} addresses the problem of analyzing the quality of facial fiducial
results using an exemplars based approach. However, several difference exist between our approaches.
Firstly, they work on a completely different problem of \emph{aggregating} fiducials from different datasets and transferring them to a target dataset through Hough based feature \emph{detection}~\cite{shenCVPR13_retrieval},
while the goal of the work presented is to \emph{select} the best locations for each fiducial among
the candidate locations provided by various candidate algorithms on every image. 
%THIS STATEMENT IS NOT CLEAR. NEED MORE CLARIFICATION.
Secondly, they use algorithms like graph matching to ensure that the detected fiducials resemble a
face~\cite{zhouICCV13_EGM}, while we either side-step such issues
(section~\ref{subsec:output_selection}) or handle them using optimization
(section~\ref{subsec:optimization}).

Recently, some promising attempts have also been made to approach the problem of facial fiducial detection in the deep-learning framework ~\cite{zhangECCV14_deepfacealign}. However, most of the proposed deep-learning based models work on low resolution images ~\cite{zhangECCV14_deepfacealign,DBLP:journals/corr/ZhangLLT14}.
This prevents us from getting accurate fiducials on actual data.
In this thesis, we present a fully-automatic
and principled approach for selecting the best fiducial location by combining results from multiple
candidate algorithms for every image.
