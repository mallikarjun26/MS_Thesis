\label{subsec:algorithm_outline}
As explained earlier, our algorithm is divided into two main sub-parts: \emph{exemplar selection}
and \emph{output selection}.
The task in exemplar selection 
is to select a subset of face images with ground truth annotations from the training dataset, 
that are representative of the \emph{variation of pose, appearance including occlusion, expression}
\etc of the dataset in consideration. Algorithm~\ref{alg:exemplar_selection} gives an outline
of our approach to exemplar selection. Note that while, we could use the entire training dataset
annotations as exemplars, it suffices to have this limited set, as we will show in
section~\ref{subsec:experimental_analysis}.

This subset of annotated images then serve as our basis for differentiating between
the various candidate algorithm outputs on any test image. The process of selecting 
the best fitting fiducials on any test image, given the exemplars, is called
output selection.
