\label{subsec:output_selection}

Once the fiducial detection of the state-of-the-art candidate algorithms are obtained for
an input image, we compute appearance 
vectors for an image patch around each fiducial location. Appearance vectors are represented 
in HOG and SIFT space. 
% Note that the SIFT feature is invariant to scale, 
% rotation and illumination which is important for the case of K-Nearest Neighbor approach. 
We concatenate these features them to form the feature vector.

We then compare these candidate algorithm feature vectors 
to the exemplars chosen from the previous approach,
and choose the candidate algorithm-exemplar image output that minimizes the sum of
euclidean distance between common features (equation~\ref{eq:main_fourth_equation}). Note that this is a simple kNN based approach,
where k=1. Alternatively, we also consider the idea of selecting individual fiducials
from various candidate algorithm outputs, to form our own facial structure that minimizes
an objective function. This is explained in the following section.
