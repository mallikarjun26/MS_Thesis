\label{sec:complementarity_sec}
% - Part based model. 
%   - Part models for part in all poses
% - Regression.
%   - Fine tuning
% - RCPR 
%   - Explicitly model occlusion as well

In this section, we discuss the reason and degree of complementarity between \textit{state-of-the-art}
methods. Methods initially proposed worked on datasets which consists of mostly frontal view faces~\cite{MaB1998} in 
a lab environment. Since the practical scenarios leads to much more complicated settings, \textit{in-the-wild} 
sort of datasets~\cite{koetsingerBFIAT11_AFLW}, \cite{artizzzuICCV13_COFW}, \cite{kumarPAMI13_faceExem} were released.
Deformable part based methods~\cite{xhuCVPR12_wild} model each part of the face separately. Also, since same part can 
look significantly different in different poses, separate model are considered. This leads to a better performance in 
profile view faces. Most of the regression based model does not consider this scenarios. Regression based methods 
tend to fairly do well in getting the accurate estimation in frontal view faces as compared to that of part based
methods. Since Artizzu \etal~\cite{artizzzuICCV13_COFW} explicitly model occlusion during training, they do well 
in the case of occluded faces. Table\ref{table:complementarity} shows the complementarity performance of various
methods. Best performing experiment is done by selecting candidate algorithm output nearest to ground truth by 
simple Euclidean metric. And average experiment was performed by averaging the estimates of all candidate algorithms.

\begin{table}%[!h]
   \centering
   \begin{tabular}{c c c c c c c c}
    \toprule[1.5pt]
     {\bf Chehra} &  {\bf Zhu} &  {\bf Intraface} &  {\bf RCPR} & {\bf Ours} & {\bf Ours} & {\bf Avg} & {\bf Best}\\
     & & & & \bf(kNN) & \bf(Opt) & & \\
    \midrule
    7.21 & 7.60 & 7.79 & 9.28 & 4.31 & 4.83 & 12.43 & 2.56 \\ 
    \bottomrule[1.5pt]
    \end{tabular}
    \caption{Table shows the failure rates of various \textit{state-of-the-art} methods and also the failure rate 
    considering the average of all the estimates and selecting the best performing fiducial among the methods for 
    COFW dataset.}
    \label{table:complementarity}
\end{table}
