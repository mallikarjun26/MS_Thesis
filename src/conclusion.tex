% Framework
% Conceptual simple
% Easily interpretable
% Efficient
% Association instead of modelling

\begin{chapquote}{Jeff Hawkins} 
The human cortex is particularly large and therefore has a massive memory
capacity. It is constantly predicting what you will see, hear, and feel, mostly in
ways you are unconscious of. These predictions are our thoughts, and, when
combined with sensory input, they are our perceptions. I call this view of the brain
the memory-prediction framework of intelligence. 

\end{chapquote}

If the machines ever to reach the capabilities of human brain, we ought to try things 
which are biologically inspired. Geared with the massive memory capability and growth 
of distributed computing, we hope that many of the computer vision problems can be 
formulated as memory association problem as compared to other traditional methods. 
Apart from conceptually simple, this sort of framework gives additional advantage of 
easy interpretability and parallelizability.

Following the above intuition, this thesis proposed exemplar based approaches for face 
fiducial detection and frontalization. Each of the method efficiently utilizes the 
information from the exemplars to achieve two different objectives. Both the approaches
are easy to interpret for any outcome. While the face fiducial detection uses both the
appearance information and structural information, face frontalization utilizes the 
strutural information of face from the exemplar database. 

\section{Summary}
We showed that our approaches in general outperform \textit{state-of-the-art} methods on 
popular datasets. We believe when the data at hand is diverse to a great extent and single 
model can not capture the diversity, it's better to go for exemplar based approaches.
With the availability of large amount of data and very good distributed frameworks, we
hope the community formulate various computer vision problems as memory association 
framework.

\section{Future Direction}
% - Complementarity 
% - Semantic fea    

In this final section, we discuss how related research fields can also benifit from 
the methods proposed in this thesis, as well as directions in which the proposed methods
can be extended for further improvement.
\begin{itemize}
    \item Meta algorithm proposed in Chapter ~\ref{ch:fid} can be extended to problems such
    as Human Pose estimation where there is enough hand annotated data to be used as exemplars
    and possible structures are fairly limited.
    \item Section ~\ref{sec:complementarity_sec} shows the quantitative analysis of degree of 
    complementarity in the various candidate algorithms for face fiducial detection. Since there 
    is huge scope for improvement, we hope this opens up new direction for further research.
    \item In the Chapter ~\ref{ch:fid}, we proposed two algorithms for fiducial detection, one which selects the entire output
    of one of the candidate algorithm and other which selects best performing parts from each of the
    candidate algorithm. Selection as whole performs better than that of by parts. Since the 
    optimization framework allows only for approximate solution, there is scope to improve accuracy by other means. 
\end{itemize}
