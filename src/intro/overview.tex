% Chapter 1
%     - Introduces to the problems 
%     - Motivation behind the problem
%     - Contributions 
% 
% Chapter 2
%     - Introduces to the concepts which are used in the thesis
%     
% Chapter 3
%     - Whole chapter dedicated to discuss the problem of face fiducial detection,
%       challenges and the two approaches proposed. We also discuss each step in detail and 
%       provide extensive experiments comparing with other state of the are methods
%       on popular datasets
% 
% Chapter 4
%     - Whole chapter dedicated to discuss the problem of face frontalization, 
%       usecases of the approach proposed. We also compare it with a recently 
%       proposed method and also provides experiments for comparison.

In this chapter, we introduced the problem this thesis addresses and also the motivation
in choosing the aforementioned problems along with the contributions.
The rest of the thesis is divided into four more chapters. Chapter 2 introduces all the
fundamental concepts briefly which are used in the thesis. Chapter 3 describes in detail
about the face fiducial detection which includes defining the problem, related work, 
approaches proposed and quantitative comparitive experiments. Similarly Chapter 4 deals
with face frontalization in detail pertaining the problem definition, previous methods
proposed, our approach, qualitative and quantitative results. And finally we end with
conclusion which illustrates the future direction of the work in Chapter 5.
