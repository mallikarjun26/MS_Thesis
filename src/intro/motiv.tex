~\citation{Chopra:2005:LSM:1068507.1068961} Computer vision as a field includes making inference out of images. There has been decades of research 
happening in this field. We also have seen fruitful applications achieved in various domains. Be it 
reconstructing of 3D world which helps in automatic car navigation, analysis of medical images for 
early detection of diseases, analysis of astronimical images to study universe in general, biometric
systems for security purposes using finger print or face images. Even though there has been decades of 
research, problems mentioned above are not completely solved as there is demand for more precision
and efficient solutions. Technological advancement from other ends such as camera capablilites,
computational capacity aids in solving problems which were not feasible earlier.

In this thesis, we are going to concentrate on approaches and applications on face images. We have seen
a lot of generic problems such as face detection, face recognition, expression detection, gaze
identification, face renactment etc currently being under research and also deployed in various 
systems. Most of the above system depends on pre-processing steps such as representation, face fiducial 
detection, tranformation etc which influences the effectiveness to a great extent. Specifically in this
thesis we consider the problem of face fiducial detection and face frontalization whose approach are 
used as pre-processing steps.

Consider a basic face recognition system. To have the representation of an individual's face as a 
model, we need to be sure the representation of face has to be aligned to get a meaningful model. 
Face fiducial detection is used as a pre-processing step to extract the features at the corresponding
locations while building the model. Similarly for an expression detection system, statistical 
distribution facial landmarks with each other helps in predecting accurate expression. Even the 
gender detection system would need facial landmarks as their initial point as fiducials are in
general differently distributed for different gender.

Other pre-processing step which we discuss in this thesis is face frontalization. Given a profile
view face, a frontal view face is synthesized. This approach helps in improving the accuracy of 
face recognition system as the profile view faces would result in degraded performance. It is
also used for face renactment where video of a person can be mimicked by replacing the face with
another.
