Computer vision is a field that includes making inference out of images and videos. There has been decades of research 
happening in this field. We have seen fruitful applications deployed in various domains. Be it 
reconstruction of 3D world, which helps in automatic car navigation, analysis of medical images for 
early detection of diseases, analysis of astronomical images to study universe in general, bio-metric
systems for security purposes using finger print or face images. Even though there has been decades of 
research effort, problems mentioned above are not completely solved as there is demand for more precision
and efficient solutions. Technological advancement from other ends such as camera capabilities,
computational capacity aids in solving problems which were not feasible earlier and also in improving 
existing solutions.

In this thesis, we concentrate on approaches and applications concerning face images. Vision community 
has been working on problems such as face detection, face recognition, expression detection, gaze
identification, face reenactment \etc., for a long time now. Also, we see these systems deployed in various 
domains for practical use~\cite{kairos},~\cite{ms_emotion}. Most of the above solutions depends on pre-processing steps such as representation, face fiducial 
detection, structural modeling \etc., which influences the effectiveness of the overall system to a great extent. 
We want to further improve the effectiveness of fiducial detection and structural representation of face 
which in-turn help in improving the accuracies of the above mentioned systems.

% Face recognition pipeline
% Problem
% Solution
Most challenging aspect of face recognition or verification system is to handle the variation in 
pose. Assume if there is method to perfectly align all the faces to one space such that the eyes,
ears, nose, mouth \etc., of all the faces fall on the same location, recognition and verification
can be done with a simple KNN approach. Schroff \etal ~\cite{conf/cvpr/SchroffKP15} relies on a purely data driven 
approach to attain in-variance to pose. This method is suitable when there is large data. 
Zhenyao \etal \cite{DBLP:journals/corr/ZhuLWT14} employ a deep network to \textit{warp} faces into a canonical 
frontal view and then learn CNN that classifies each face. Sun \etal~\cite{NIPS2014_5416} relies on 
extracting features from different face patches to counter pose variation. Facial fiducials help in
aligned feature representation as we could extract feature corresponding to various key points on 
the face independent of overall pose and hence achieve invariability to pose. 

Facial expression recognition systems such as Chew \etal \cite{conf/fgr/ChewLLSCS11}, Valstar \etal \cite{Valstar:2006:FAF:1153172.1153856} 
use face fiducial detection as their starting point of the system. Statistical distribution facial 
landmarks with each other helps in predicting accurate expression. Similarly the gender detection system 
would need facial landmarks as their initial point as fiducials are differently distributed 
for different gender in general.

% Consider a basic face recognition system. To have the representation of an individual's face as a 
% model, we need to be sure the representation of face has to be aligned to get a meaningful model. 
% Face fiducial detection is used as a pre-processing step to extract the features at the corresponding
% locations while building the model. 

We also address the problem of face frontalization. This approach helps in improving the accuracy of 
face recognition system. Data is expensive. To obtain all possible poses of a given person to 
come up with a model for recognition system is a tedious task. Models generated with fewer images
of the person which predominantly includes frontal pose faces wouldn\textquotesingle t perform well on profile
view test image. Approach proposed for face frontalization can also be used for face reenactment where 
video of a person can be mimicked by replacing the face of another person.
