% fid
% Our approach attempts the problem of fiducial detec-
% tion as a classification problem of differentiating be-
% tween the best vs the rest among fiducial detection
% outputs of state-of-the-art algorithms. To our knowl-
% edge, this is the first time such an approach has been
% attempted.
% • Since we only focus on selecting from a variety of so-
% lution candidates, this allows our pre-processing rou-
% tine to generate outputs corresponding to a variety
% of face detector initialization, thus rendering our al-
% gorithm insensitive to initialization unlike other ap-
% proaches.
% • Combining approaches better geared for sub-pixel ac-
% curacy and algorithms designed for robustness leads
% to our approach outperforming state-of-the-art in both
% accuracy and robustness.
% 
% front


In this thesis, we propose exemplar based approaches for two fundamental problems
related to face images. In both the cases, we provide extensive experimental 
analysis to show that the proposed approaches perform superior to the 
state-of-the-art methods on popular datasets. Face fiducial detection approach 
manifests as two algorithms, one based on optimizing an objective function with 
quadratic terms (refer to Section~\ref{subsec:optimization}) and the other based on simple KNN(refer to Section~\ref{subsec:output_selection}). Proposed face frontalization
approach can be used either as a pre-processing step in face recognition, gender
identification algorithms or also in rendering face video for face reenactment.

Proposed face fiducial is \textit{initialization-insensitive, pose/occlusion and
expression-robust} approach with the following characteristics,
\begin{itemize}
\item{Our approach attempts the problem of fiducial detection 
as a classification problem of differentiating between 
the best vs the rest among fiducial detection outputs of 
state-of-the-art algorithms. To our knowledge, this is the 
first time such an approach has been attempted.}
\item{Since we only focus on selecting from a variety of solution 
candidates, this allows our pre-processing routine 
to generate outputs corresponding to a variety of face 
detector initialization, thus rendering our algorithm 
insensitive to initialization unlike other approaches.}
\item{Combining approaches better geared for sub-pixel accuracy 
and algorithms designed for robustness leads to our 
approach outperforming state-of-the-art in both accuracy 
and robustness.}
\end{itemize}

We compare our approach with five of \textit{state-of-the-art} methods on
three popular datasets such as LFPW, COFW and AFLW. In some cases, we
report as much as $17$\% improvement in the accuracy.
\newline
For face frontalization, we employ an exemplar based approach to find the 
transformation that relates the profile view to the frontal view, and use it 
to generate realistic frontalizations. In specific,
\begin{itemize}
\item{Our method does not involve estimating 3D model of the face,
which is a common approach in previous work in this area. This
leads to an efficient solution, since we avoid the complexity of
adding one more dimension to the problem}
\item{Our method also retains the structural information of the individual 
as compared to that of a recent method, which assumes a generic 3D model
for synthesis}
\end{itemize}

We compare our approach with a recent \textit{state-of-the-art} method. We
provide qualitative comparison on various faces extracted from the videos
available online. We also provide quantitative result on a face recognition 
dataset by frontalizing all the faces before the recognition task and show 
that our method performs significantly better and efficient.  
