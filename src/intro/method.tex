“Represent all the data with a nonparametric model rather than trying to
summarize it with a parametric model, because with very large data sources, the
data holds a lot of detail... Now go out and gather some data, and see what it can
do.”
Alon Halevy, Peter Norvig, and Fernando Pereira,
“The Unreasonable Effectiveness of Data” (Google, 2009)

The above statement is more so true when the data at hand is diverse at many fronts. In our case, 
the data is the face images which can have variations in terms of age, gender, expression, pose, 
facial structure etc. Trying to come up with a parametric model which holds all these information
leads to poorly performing systems. Also whenever there is new data, the model has to be updated
which is not so efficient. Rather its better to go with the data driven solutions.

For face fiducial detection, we employ exemplar based approach to to select the best solution 
from among outputs of regression and mixture of trees based algorithms (which we call candidate 
algorithms). We show that by using a very simple SIFT and HOG based descriptor, it is possible to 
identify the most accurate fiducial outputs from a set of results produced by candidate algorithms 
on any given test image. 

For face frontalization We employ an exemplar based approach to find the transformation that relates 
the profile view to the frontal view, and use it to generate realistic frontalizations. Our method 
does not involve estimating 3D model of the face, which is a common approach in previous work in 
this area. This leads to an efficient solution, since we avoid the complexity of adding one more 
dimension to the problem. 
